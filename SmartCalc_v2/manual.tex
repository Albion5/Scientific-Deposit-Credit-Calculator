\documentclass{article}
\usepackage{geometry}
\usepackage{indentfirst}
\geometry{verbose,a4paper,tmargin=2cm,bmargin=2cm,lmargin=2.5cm,rmargin=1.5cm}
\title{Quick guide to the SmartCalc v2.0}

\author{Erikur Sirrus}

\date{February 2024}

\begin{document}

\maketitle
\pagebreak


\section{Introduction}
The SmartCalc v2.0 is an extended version of the usual calculator, which can be found in the standard applications of each operating system. The Core was implemented in the C programming language using structured programming. The GUI was implemented using the Qt library. In addition to basic arithmetic operations such as add/subtract and multiply/divide, this calculator was supplemented with the ability to calculate arithmetic expressions by following the order, as well as some mathematical functions (sine, cosine, logarithm, etc.).

Besides calculating expressions, it also supports the use of the X variable and the graphing of the corresponding function.

\section{Arithmetic operations and mathematical functions}

The following arithmetic operations and mathematical functions are supported:
  \begin{itemize}
    \item Arithmetic operators
    \begin{itemize}
      \item Brackets -- (a + b)
      \item Addition -- a + b
      \item Subtraction -- a - b
      \item Multiplication -- a * b
      \item Division -- a / b
      \item Power -- a \^\ b
      \item Modulus -- a mod b
      \item Unary plus -- +a
      \item Unary minus -- -a
    \end{itemize}
    \item Functions
    \begin{itemize}
      \item Cosine calculation -- cos(x)
      \item Sine calculation -- sin(x)
      \item Tangent calculation -- tan(x)
      \item Arc cosine calculation -- acos(x)
      \item Arc sine calculation -- asin(x)
      \item Arc tangent calculation -- atan(x)
      \item Square root calculation -- sqrt(x)
      \item Natural logarithm calculation -- ln(x)
      \item Common logarithm calculation -- log(x)
      \item Absolute value calculation -- abs(x)
    \end{itemize}
  \end{itemize}

\section{Plotting}

There is implemented plotting a graph of a function given by an expression in infix notation with the variable X (with coordinate axes, mark of the used scale and an adaptive grid).
  
\end{document}

